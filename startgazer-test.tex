% Options for packages loaded elsewhere
\PassOptionsToPackage{unicode}{hyperref}
\PassOptionsToPackage{hyphens}{url}
%
\documentclass[
]{article}
\usepackage{amsmath,amssymb}
\usepackage{lmodern}
\usepackage{iftex}
\ifPDFTeX
  \usepackage[T1]{fontenc}
  \usepackage[utf8]{inputenc}
  \usepackage{textcomp} % provide euro and other symbols
\else % if luatex or xetex
  \usepackage{unicode-math}
  \defaultfontfeatures{Scale=MatchLowercase}
  \defaultfontfeatures[\rmfamily]{Ligatures=TeX,Scale=1}
\fi
% Use upquote if available, for straight quotes in verbatim environments
\IfFileExists{upquote.sty}{\usepackage{upquote}}{}
\IfFileExists{microtype.sty}{% use microtype if available
  \usepackage[]{microtype}
  \UseMicrotypeSet[protrusion]{basicmath} % disable protrusion for tt fonts
}{}
\makeatletter
\@ifundefined{KOMAClassName}{% if non-KOMA class
  \IfFileExists{parskip.sty}{%
    \usepackage{parskip}
  }{% else
    \setlength{\parindent}{0pt}
    \setlength{\parskip}{6pt plus 2pt minus 1pt}}
}{% if KOMA class
  \KOMAoptions{parskip=half}}
\makeatother
\usepackage{xcolor}
\IfFileExists{xurl.sty}{\usepackage{xurl}}{} % add URL line breaks if available
\IfFileExists{bookmark.sty}{\usepackage{bookmark}}{\usepackage{hyperref}}
\hypersetup{
  hidelinks,
  pdfcreator={LaTeX via pandoc}}
\urlstyle{same} % disable monospaced font for URLs
\usepackage[margin=1in]{geometry}
\usepackage{graphicx}
\makeatletter
\def\maxwidth{\ifdim\Gin@nat@width>\linewidth\linewidth\else\Gin@nat@width\fi}
\def\maxheight{\ifdim\Gin@nat@height>\textheight\textheight\else\Gin@nat@height\fi}
\makeatother
% Scale images if necessary, so that they will not overflow the page
% margins by default, and it is still possible to overwrite the defaults
% using explicit options in \includegraphics[width, height, ...]{}
\setkeys{Gin}{width=\maxwidth,height=\maxheight,keepaspectratio}
% Set default figure placement to htbp
\makeatletter
\def\fps@figure{htbp}
\makeatother
\setlength{\emergencystretch}{3em} % prevent overfull lines
\providecommand{\tightlist}{%
  \setlength{\itemsep}{0pt}\setlength{\parskip}{0pt}}
\setcounter{secnumdepth}{-\maxdimen} % remove section numbering
\usepackage{booktabs}
\usepackage{longtable}
\usepackage{array}
\usepackage{multirow}
\usepackage{wrapfig}
\usepackage{float}
\usepackage{colortbl}
\usepackage{pdflscape}
\usepackage{tabu}
\usepackage{threeparttable}
\usepackage{threeparttablex}
\usepackage[normalem]{ulem}
\usepackage{makecell}
\usepackage{xcolor}
\ifLuaTeX
  \usepackage{selnolig}  % disable illegal ligatures
\fi

\author{}
\date{\vspace{-2.5em}}

\begin{document}

\hypertarget{changes-in-the-persistence-of-the-cpi-and-the-ppi}{%
\section{Changes in the persistence of the CPI and the
PPI}\label{changes-in-the-persistence-of-the-cpi-and-the-ppi}}

The data set is Consumer price inflation (CPI) all items and Producer
price inflation(PPI) for the UK from March 1997 to November 2019. Figure
@ref(fig:Annualinflation) and @ref(fig:acd) plot the UK annual and
monthly CPI and PPI. To analyze persistent changes in CPI and PPI, the
first step is to determine their basic time series properties.
Determining the sequence of data integration is of particular
importance. That is, evaluating whether the PPI and CPI inflation rates
are I (0) processes. If the inflation rate is a non-stationary I(1)
process, then the price level will be an I(2) process, so the analysis
to determine the pass-through of producer price shocks to consumer
prices will be more complex.

\begin{figure}
\centering
\includegraphics{startgazer-test_files/figure-latex/Annualinflation-1.pdf}
\caption{UK Annual inflation: CPI vs PPI}
\end{figure}

\begin{figure}
\centering
\includegraphics{startgazer-test_files/figure-latex/acd-1.pdf}
\caption{UK Monthly CPI and PPI inflation}
\end{figure}

Another test we can conduct is the Augmented Dickey--Fuller (ADF)
t-statistic test to find if the series has a unit root (a series with a
trend line will have a unit root and result in a large p-value). ADF
tests reject the null of non-stationarity of both PPI and CPI. The test
identifies all stationary periods within the sample. The period
identified as I(0) can be analysed according to the timing and operating
rules of monetary policy.

\begin{verbatim}
## 
##  Augmented Dickey-Fuller Test
## 
## data:  monthly_ppi_cpi$cpi
## Dickey-Fuller = -5.2699, Lag order = 6, p-value = 0.01
## alternative hypothesis: stationary
\end{verbatim}

\begin{verbatim}
## 
##  Augmented Dickey-Fuller Test
## 
## data:  monthly_ppi_cpi$ppi
## Dickey-Fuller = -6.1957, Lag order = 6, p-value = 0.01
## alternative hypothesis: stationary
\end{verbatim}

\newpage

\hypertarget{methodology-to-evaluate-the-predictive-content-of-the-ppi-for-the-cpi}{%
\section{Methodology to evaluate the predictive content of the PPI for
the
CPI}\label{methodology-to-evaluate-the-predictive-content-of-the-ppi-for-the-cpi}}

In order to study the forecasting ability of PPI inflation to CPI
inflation, the related equation is:

\[\pi_{t}^{C P I}=\mu_{0}+\sum_{j=1}^{p} \alpha_{j} \pi_{t-j}^{C P I}+\sum_{j=1}^{p} \beta_{j} \pi_{t-j}^{P P I}+\varepsilon_{t}\]

where \(\varepsilon_{t}\) is considered as white noise. The model is
typically estimated by ordinary least squares (OLS), and the number of
lags, p, is usually determined by using an information criterion such as
the Bayesian information criterion (BIC). We decide p as 12 in this
case.

\newpage

\begin{table}[!htbp] \centering 
  \caption{} 
  \label{} 
\begin{tabular}{@{\extracolsep{5pt}}lD{.}{.}{-2} } 
\\[-1.8ex]\hline 
\hline \\[-1.8ex] 
 & \multicolumn{1}{c}{\textit{Dependent variable:}} \\ 
\cline{2-2} 
\\[-1.8ex] & \multicolumn{1}{c}{COPY[, 1]} \\ 
 & \multicolumn{1}{c}{CPI ALL ITEMS} \\ 
\hline \\[-1.8ex] 
 CPI\_lag1 & -0.13^{**} \\ 
  CPI\_lag2 & -0.08 \\ 
  CPI\_lag3 & -0.05 \\ 
  CPI\_lag4 & -0.05 \\ 
  CPI\_lag5 & -0.05 \\ 
  CPI\_lag6 & 0.10^{*} \\ 
  CPI\_lag7 & 0.005 \\ 
  CPI\_lag8 & -0.02 \\ 
  CPI\_lag9 & -0.06 \\ 
  CPI\_lag10 & -0.07 \\ 
  CPI\_lag11 & -0.05 \\ 
  CPI\_lag12 & 0.66^{**} \\ 
  ppi2 & 0.12^{**} \\ 
  ppi3 & -0.07 \\ 
  ppi4 & -0.03 \\ 
  ppi5 & -0.003 \\ 
  ppi6 & 0.01 \\ 
  ppi7 & -0.05 \\ 
  ppi8 & 0.004 \\ 
  ppi9 & 0.02 \\ 
  ppi10 & 0.05 \\ 
  ppi11 & -0.001 \\ 
  ppi12 & -0.05 \\ 
  ppi13 & 0.02 \\ 
  Constant & 0.09^{**} \\ 
 \hline \\[-1.8ex] 
Observations & \multicolumn{1}{c}{273} \\ 
R$^{2}$ & \multicolumn{1}{c}{0.65} \\ 
Adjusted R$^{2}$ & \multicolumn{1}{c}{0.62} \\ 
Residual Std. Error & \multicolumn{1}{c}{0.22} \\ 
F Statistic & \multicolumn{1}{c}{19.60$^{**}$} \\ 
\hline 
\hline \\[-1.8ex] 
\textit{Note:}  & \multicolumn{1}{r}{$^{*}$p$<$0.05; $^{**}$p$<$0.01; $^{***}$p$<$[0.***]} \\ 
\end{tabular} 
\end{table}

\end{document}
