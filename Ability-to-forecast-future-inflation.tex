% Options for packages loaded elsewhere
\PassOptionsToPackage{unicode}{hyperref}
\PassOptionsToPackage{hyphens}{url}
%
\documentclass[
]{article}
\usepackage{amsmath,amssymb}
\usepackage{lmodern}
\usepackage{iftex}
\ifPDFTeX
  \usepackage[T1]{fontenc}
  \usepackage[utf8]{inputenc}
  \usepackage{textcomp} % provide euro and other symbols
\else % if luatex or xetex
  \usepackage{unicode-math}
  \defaultfontfeatures{Scale=MatchLowercase}
  \defaultfontfeatures[\rmfamily]{Ligatures=TeX,Scale=1}
\fi
% Use upquote if available, for straight quotes in verbatim environments
\IfFileExists{upquote.sty}{\usepackage{upquote}}{}
\IfFileExists{microtype.sty}{% use microtype if available
  \usepackage[]{microtype}
  \UseMicrotypeSet[protrusion]{basicmath} % disable protrusion for tt fonts
}{}
\makeatletter
\@ifundefined{KOMAClassName}{% if non-KOMA class
  \IfFileExists{parskip.sty}{%
    \usepackage{parskip}
  }{% else
    \setlength{\parindent}{0pt}
    \setlength{\parskip}{6pt plus 2pt minus 1pt}}
}{% if KOMA class
  \KOMAoptions{parskip=half}}
\makeatother
\usepackage{xcolor}
\IfFileExists{xurl.sty}{\usepackage{xurl}}{} % add URL line breaks if available
\IfFileExists{bookmark.sty}{\usepackage{bookmark}}{\usepackage{hyperref}}
\hypersetup{
  pdftitle={EVALUATING EXCLUDING FOOD AND ENERGY CORE INFLATION MEASURES},
  pdfauthor={Yiyi},
  hidelinks,
  pdfcreator={LaTeX via pandoc}}
\urlstyle{same} % disable monospaced font for URLs
\usepackage[margin=1in]{geometry}
\usepackage{graphicx}
\makeatletter
\def\maxwidth{\ifdim\Gin@nat@width>\linewidth\linewidth\else\Gin@nat@width\fi}
\def\maxheight{\ifdim\Gin@nat@height>\textheight\textheight\else\Gin@nat@height\fi}
\makeatother
% Scale images if necessary, so that they will not overflow the page
% margins by default, and it is still possible to overwrite the defaults
% using explicit options in \includegraphics[width, height, ...]{}
\setkeys{Gin}{width=\maxwidth,height=\maxheight,keepaspectratio}
% Set default figure placement to htbp
\makeatletter
\def\fps@figure{htbp}
\makeatother
\setlength{\emergencystretch}{3em} % prevent overfull lines
\providecommand{\tightlist}{%
  \setlength{\itemsep}{0pt}\setlength{\parskip}{0pt}}
\setcounter{secnumdepth}{-\maxdimen} % remove section numbering
\usepackage{booktabs}
\usepackage{longtable}
\usepackage{array}
\usepackage{multirow}
\usepackage{wrapfig}
\usepackage{float}
\usepackage{colortbl}
\usepackage{pdflscape}
\usepackage{tabu}
\usepackage{threeparttable}
\usepackage{threeparttablex}
\usepackage[normalem]{ulem}
\usepackage{makecell}
\usepackage{xcolor}
\ifLuaTeX
  \usepackage{selnolig}  % disable illegal ligatures
\fi

\title{EVALUATING EXCLUDING FOOD AND ENERGY CORE INFLATION MEASURES}
\author{Yiyi}
\date{2021-12-15}

\begin{document}
\maketitle

\hypertarget{criteria-for-evaluation-forecasting-future-inflation}{%
\section{Criteria for evaluation : forecasting future
inflation}\label{criteria-for-evaluation-forecasting-future-inflation}}

This work uses Cogley's model to test the forecasting ability of core
inflation rate, and compares the forecasting ability of excluding food
and energy core inflation rates.~Cogley's model is based on Bryan and
Cecchetti's definition of core inflation: ``Core inflation is changes in
the price level that are expected to persist over a long period of
time.''

According to Bryan and Cecchetti's definition, a valid core inflation is
one that is ``pure'' after removing temporary factors from measured real
inflation. On this basis, Cogley developed the following model to
evaluate the predictive power of core inflation:

\[\pi_{t+h}-\pi_{t}=\alpha_{h}+\beta_{h}\left(\pi_{t}-\pi_{t}^{c}\right)+u_{t+h}\]

Here, x represents the headline inflation rate and core x represents
some core inflation indicator, both year-on-year data. Parameter h is
6,9 and 12 (month) ahead in this case. For sufficiently large H, the
core deviation, \(\left(\pi_{t}-\pi_{t}^{c}\right)\), should be
inversely related to subsequent changes in inflation,
\(\pi_{t+h}-\pi_{t}\). Moreover, in order for the candidate to satisfy
equation (1), the coefficients in the regression, should satisfy
\(\alpha=0\) and \(\beta = -1\).

Of importance to the forecasting model is the estimated coefficient of
\(\beta\), which indicates whether core inflation has sufficiently
purified the transitory component. Because if the absolute value of the
estimated coefficient is equal to 1, it indicates that the model is a
random walk process, and the components removed from the core inflation
do not contain any information that predicts future overall inflation.
If \(\beta=-1\), the forecasting capacity for core inflation is the best
. This proves that core inflation has fully captured the trend
components of overall inflation and has a complete forecasting ability
for future inflation.

\begin{enumerate}
\def\labelenumi{\arabic{enumi}.}
\item
  If the \(|\beta|<1\), it indicates that subsequent changes in
  inflation are overestimated;
\item
  If the \(|\beta|>1\), it shows that underestimation of the current
  temporary movement in headline inflation.
\end{enumerate}

Therefore, the closer the absolute value of the estimated regression
coefficient \(\beta\) is to 1, the better the predictive power of core
inflation is. In addition, the root mean square error RMSE
\(=\sqrt{\frac{1}{\mathrm{~T}} \sum_{\mathrm{t}=1}^{\mathrm{T}}\left(\pi_{\mathrm{t}}-\hat\pi_{\mathrm{t}}\right)^{2}}\)obtained
by Cogley regression represents the deviation between the predicted
value and the actual value. \(\hat\pi_{\mathrm{t}}\) is the forecast
value of the inflation rate. The smaller the RMSE, the more accurate the
forecast. and the better the forecast of core inflation.

\#Trimmed Mean

\begin{table}
\centering
\begin{tabular}[t]{l|r|r|r|r|r|r}
\hline
\multicolumn{1}{c|}{ } & \multicolumn{6}{c}{Core Inflation 6} \\
\cline{2-7}
\multicolumn{1}{c|}{ } & \multicolumn{3}{c|}{Excluding Food and Energy} & \multicolumn{3}{c}{Trimmed Mean} \\
\cline{2-4} \cline{5-7}
  & RMSE & \$\textbackslash{}alpha\$ & \$\textbackslash{}beta\$ & RMSE & \$\textbackslash{}alpha\$ & \$\textbackslash{}beta\$\\
\hline
1 months ahead & 0.2875491 & 0.0084757 & -0.0195335 & 0.2866793 & 0.0645846 & -0.0510096\\
\hline
2 months ahead & 0.4433265 & 0.0284204 & -0.0724343 & 0.4430588 & 0.1329443 & -0.1059460\\
\hline
3 months ahead & 0.5778391 & 0.0552011 & -0.1493606 & 0.5791695 & 0.2423759 & -0.1956752\\
\hline
4 months ahead & 0.6936614 & 0.0786384 & -0.2277337 & 0.7019321 & 0.2963575 & -0.2439679\\
\hline
5 months ahead & 0.7942037 & 0.0995841 & -0.3039190 & 0.8085734 & 0.3722336 & -0.3111952\\
\hline
6 months ahead & 0.8797910 & 0.1218862 & -0.3837018 & 0.8997629 & 0.4729218 & -0.3983625\\
\hline
7 months ahead & 0.9466686 & 0.1433650 & -0.4598274 & 0.9683200 & 0.6086117 & -0.5132120\\
\hline
8 months ahead & 1.0048770 & 0.1599479 & -0.5362292 & 1.0338213 & 0.6923265 & -0.5903104\\
\hline
9 months ahead & 1.0531805 & 0.1748061 & -0.6051095 & 1.0921331 & 0.7459039 & -0.6422965\\
\hline
10 months ahead & 1.0900066 & 0.1942808 & -0.6803401 & 1.1417008 & 0.8047791 & -0.6967543\\
\hline
11 months ahead & 1.1239888 & 0.2127016 & -0.7557223 & 1.1962377 & 0.8123711 & -0.7108918\\
\hline
12 months ahead & 1.1554869 & 0.2285637 & -0.8169504 & 1.2412841 & 0.8471592 & -0.7446529\\
\hline
\end{tabular}
\end{table}

\hypertarget{results}{%
\section{Results}\label{results}}

From the above discussion, I used my own data to re-simulate the values
of the parameters.The results please see Table 1.

\hypertarget{rmse}{%
\subsection{RMSE}\label{rmse}}

\begin{table}[h]
\centering
\caption{Forecasting future inflation}
\label{tab:my-table}
\begin{tabular}{|c|cll|}
\hline
H                       & \multicolumn{3}{c|}{Excluding Food and Energy}                                               \\ \hline
                        & \multicolumn{1}{c|}{a}       & \multicolumn{1}{c|}{b}        & \multicolumn{1}{c|}{RMSE}     \\ \hline
\multicolumn{1}{|l|}{6} & \multicolumn{1}{l|}{0.12189} & \multicolumn{1}{l|}{-0.38370} & 0.879791                      \\ \hline
\multicolumn{1}{|l|}{9} & \multicolumn{1}{l|}{0.17481} & \multicolumn{1}{l|}{-0.60511} & 1.053181                      \\ \hline
12                      & \multicolumn{1}{c|}{0.229}   & \multicolumn{1}{c|}{-0.81695} & \multicolumn{1}{c|}{1.155487} \\ \hline
\end{tabular}
\end{table}

\hypertarget{conclusions}{%
\section{Conclusions}\label{conclusions}}

As expected, all Cogley regressions of the core inflation rate with
\(\beta\) estimates are all negative. The beta keeps falling as the
number of forecast periods rises; The forecast RMSE for core inflation
rates increases as the number of forecast periods increases
continuously. This suggests that the longer the forecast period, the
more inaccurate the forecast.

\end{document}
