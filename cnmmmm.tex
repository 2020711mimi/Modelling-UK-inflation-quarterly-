% Options for packages loaded elsewhere
\PassOptionsToPackage{unicode}{hyperref}
\PassOptionsToPackage{hyphens}{url}
%
\documentclass[
]{article}
\usepackage{amsmath,amssymb}
\usepackage{lmodern}
\usepackage{iftex}
\ifPDFTeX
  \usepackage[T1]{fontenc}
  \usepackage[utf8]{inputenc}
  \usepackage{textcomp} % provide euro and other symbols
\else % if luatex or xetex
  \usepackage{unicode-math}
  \defaultfontfeatures{Scale=MatchLowercase}
  \defaultfontfeatures[\rmfamily]{Ligatures=TeX,Scale=1}
\fi
% Use upquote if available, for straight quotes in verbatim environments
\IfFileExists{upquote.sty}{\usepackage{upquote}}{}
\IfFileExists{microtype.sty}{% use microtype if available
  \usepackage[]{microtype}
  \UseMicrotypeSet[protrusion]{basicmath} % disable protrusion for tt fonts
}{}
\makeatletter
\@ifundefined{KOMAClassName}{% if non-KOMA class
  \IfFileExists{parskip.sty}{%
    \usepackage{parskip}
  }{% else
    \setlength{\parindent}{0pt}
    \setlength{\parskip}{6pt plus 2pt minus 1pt}}
}{% if KOMA class
  \KOMAoptions{parskip=half}}
\makeatother
\usepackage{xcolor}
\IfFileExists{xurl.sty}{\usepackage{xurl}}{} % add URL line breaks if available
\IfFileExists{bookmark.sty}{\usepackage{bookmark}}{\usepackage{hyperref}}
\hypersetup{
  hidelinks,
  pdfcreator={LaTeX via pandoc}}
\urlstyle{same} % disable monospaced font for URLs
\usepackage[margin=1in]{geometry}
\usepackage{color}
\usepackage{fancyvrb}
\newcommand{\VerbBar}{|}
\newcommand{\VERB}{\Verb[commandchars=\\\{\}]}
\DefineVerbatimEnvironment{Highlighting}{Verbatim}{commandchars=\\\{\}}
% Add ',fontsize=\small' for more characters per line
\usepackage{framed}
\definecolor{shadecolor}{RGB}{248,248,248}
\newenvironment{Shaded}{\begin{snugshade}}{\end{snugshade}}
\newcommand{\AlertTok}[1]{\textcolor[rgb]{0.94,0.16,0.16}{#1}}
\newcommand{\AnnotationTok}[1]{\textcolor[rgb]{0.56,0.35,0.01}{\textbf{\textit{#1}}}}
\newcommand{\AttributeTok}[1]{\textcolor[rgb]{0.77,0.63,0.00}{#1}}
\newcommand{\BaseNTok}[1]{\textcolor[rgb]{0.00,0.00,0.81}{#1}}
\newcommand{\BuiltInTok}[1]{#1}
\newcommand{\CharTok}[1]{\textcolor[rgb]{0.31,0.60,0.02}{#1}}
\newcommand{\CommentTok}[1]{\textcolor[rgb]{0.56,0.35,0.01}{\textit{#1}}}
\newcommand{\CommentVarTok}[1]{\textcolor[rgb]{0.56,0.35,0.01}{\textbf{\textit{#1}}}}
\newcommand{\ConstantTok}[1]{\textcolor[rgb]{0.00,0.00,0.00}{#1}}
\newcommand{\ControlFlowTok}[1]{\textcolor[rgb]{0.13,0.29,0.53}{\textbf{#1}}}
\newcommand{\DataTypeTok}[1]{\textcolor[rgb]{0.13,0.29,0.53}{#1}}
\newcommand{\DecValTok}[1]{\textcolor[rgb]{0.00,0.00,0.81}{#1}}
\newcommand{\DocumentationTok}[1]{\textcolor[rgb]{0.56,0.35,0.01}{\textbf{\textit{#1}}}}
\newcommand{\ErrorTok}[1]{\textcolor[rgb]{0.64,0.00,0.00}{\textbf{#1}}}
\newcommand{\ExtensionTok}[1]{#1}
\newcommand{\FloatTok}[1]{\textcolor[rgb]{0.00,0.00,0.81}{#1}}
\newcommand{\FunctionTok}[1]{\textcolor[rgb]{0.00,0.00,0.00}{#1}}
\newcommand{\ImportTok}[1]{#1}
\newcommand{\InformationTok}[1]{\textcolor[rgb]{0.56,0.35,0.01}{\textbf{\textit{#1}}}}
\newcommand{\KeywordTok}[1]{\textcolor[rgb]{0.13,0.29,0.53}{\textbf{#1}}}
\newcommand{\NormalTok}[1]{#1}
\newcommand{\OperatorTok}[1]{\textcolor[rgb]{0.81,0.36,0.00}{\textbf{#1}}}
\newcommand{\OtherTok}[1]{\textcolor[rgb]{0.56,0.35,0.01}{#1}}
\newcommand{\PreprocessorTok}[1]{\textcolor[rgb]{0.56,0.35,0.01}{\textit{#1}}}
\newcommand{\RegionMarkerTok}[1]{#1}
\newcommand{\SpecialCharTok}[1]{\textcolor[rgb]{0.00,0.00,0.00}{#1}}
\newcommand{\SpecialStringTok}[1]{\textcolor[rgb]{0.31,0.60,0.02}{#1}}
\newcommand{\StringTok}[1]{\textcolor[rgb]{0.31,0.60,0.02}{#1}}
\newcommand{\VariableTok}[1]{\textcolor[rgb]{0.00,0.00,0.00}{#1}}
\newcommand{\VerbatimStringTok}[1]{\textcolor[rgb]{0.31,0.60,0.02}{#1}}
\newcommand{\WarningTok}[1]{\textcolor[rgb]{0.56,0.35,0.01}{\textbf{\textit{#1}}}}
\usepackage{graphicx}
\makeatletter
\def\maxwidth{\ifdim\Gin@nat@width>\linewidth\linewidth\else\Gin@nat@width\fi}
\def\maxheight{\ifdim\Gin@nat@height>\textheight\textheight\else\Gin@nat@height\fi}
\makeatother
% Scale images if necessary, so that they will not overflow the page
% margins by default, and it is still possible to overwrite the defaults
% using explicit options in \includegraphics[width, height, ...]{}
\setkeys{Gin}{width=\maxwidth,height=\maxheight,keepaspectratio}
% Set default figure placement to htbp
\makeatletter
\def\fps@figure{htbp}
\makeatother
\setlength{\emergencystretch}{3em} % prevent overfull lines
\providecommand{\tightlist}{%
  \setlength{\itemsep}{0pt}\setlength{\parskip}{0pt}}
\setcounter{secnumdepth}{-\maxdimen} % remove section numbering
\ifLuaTeX
  \usepackage{selnolig}  % disable illegal ligatures
\fi

\author{}
\date{\vspace{-2.5em}}

\begin{document}

\hypertarget{data}{%
\section{Data}\label{data}}

\hypertarget{import}{%
\subsection{Import}\label{import}}

\begin{Shaded}
\begin{Highlighting}[]
\NormalTok{data }\OtherTok{\textless{}{-}} \FunctionTok{read.csv}\NormalTok{(}\StringTok{"data.csv"}\NormalTok{)}
\FunctionTok{head}\NormalTok{(data)}
\end{Highlighting}
\end{Shaded}

\begin{verbatim}
##   X speed dist
## 1 1     4    2
## 2 2     4   10
## 3 3     7    4
## 4 4     7   22
## 5 5     8   16
## 6 6     9   10
\end{verbatim}

\hypertarget{putting-your-entire-data-into-the-.rmd-file}{%
\subsection{Putting your entire data into the .rmd
file}\label{putting-your-entire-data-into-the-.rmd-file}}

Applying the function \texttt{dput()} to an object gives you the code
needed to reproduce that object. So you could paste that code into your
\texttt{.rmd} file if you don't want to have extra data files. This
makes sense were data files are small.

\begin{Shaded}
\begin{Highlighting}[]
\FunctionTok{dput}\NormalTok{(data)}
\end{Highlighting}
\end{Shaded}

\begin{verbatim}
## structure(list(X = 1:50, speed = c(4L, 4L, 7L, 7L, 8L, 9L, 10L, 
## 10L, 10L, 11L, 11L, 12L, 12L, 12L, 12L, 13L, 13L, 13L, 13L, 14L, 
## 14L, 14L, 14L, 15L, 15L, 15L, 16L, 16L, 17L, 17L, 17L, 18L, 18L, 
## 18L, 18L, 19L, 19L, 19L, 20L, 20L, 20L, 20L, 20L, 22L, 23L, 24L, 
## 24L, 24L, 24L, 25L), dist = c(2L, 10L, 4L, 22L, 16L, 10L, 18L, 
## 26L, 34L, 17L, 28L, 14L, 20L, 24L, 28L, 26L, 34L, 34L, 46L, 26L, 
## 36L, 60L, 80L, 20L, 26L, 54L, 32L, 40L, 32L, 40L, 50L, 42L, 56L, 
## 76L, 84L, 36L, 46L, 68L, 32L, 48L, 52L, 56L, 64L, 66L, 54L, 70L, 
## 92L, 93L, 120L, 85L)), class = "data.frame", row.names = c(NA, 
## -50L))
\end{verbatim}

You can then insert the dput output in your \texttt{.rmd} as below.

\begin{Shaded}
\begin{Highlighting}[]
\NormalTok{data }\OtherTok{\textless{}{-}} \FunctionTok{structure}\NormalTok{(}\FunctionTok{list}\NormalTok{(}\AttributeTok{X =} \DecValTok{1}\SpecialCharTok{:}\DecValTok{50}\NormalTok{, }\AttributeTok{speed =} \FunctionTok{c}\NormalTok{(4L, 4L, 7L, 7L, 8L, 9L, 10L, }
\NormalTok{10L, 10L, 11L, 11L, 12L, 12L, 12L, 12L, 13L, 13L, 13L, 13L, 14L, }
\NormalTok{14L, 14L, 14L, 15L, 15L, 15L, 16L, 16L, 17L, 17L, 17L, 18L, 18L, }
\NormalTok{18L, 18L, 19L, 19L, 19L, 20L, 20L, 20L, 20L, 20L, 22L, 23L, 24L, }
\NormalTok{24L, 24L, 24L, 25L), }\AttributeTok{dist =} \FunctionTok{c}\NormalTok{(2L, 10L, 4L, 22L, 16L, 10L, 18L, }
\NormalTok{26L, 34L, 17L, 28L, 14L, 20L, 24L, 28L, 26L, 34L, 34L, 46L, 26L, }
\NormalTok{36L, 60L, 80L, 20L, 26L, 54L, 32L, 40L, 32L, 40L, 50L, 42L, 56L, }
\NormalTok{76L, 84L, 36L, 46L, 68L, 32L, 48L, 52L, 56L, 64L, 66L, 54L, 70L, }
\NormalTok{92L, 93L, 120L, 85L)), }
\AttributeTok{class =} \StringTok{"data.frame"}\NormalTok{, }\AttributeTok{row.names =} \FunctionTok{c}\NormalTok{(}\ConstantTok{NA}\NormalTok{, }
\SpecialCharTok{{-}}\NormalTok{50L))}
\end{Highlighting}
\end{Shaded}

\hypertarget{sec:tables}{%
\section{Tables}\label{sec:tables}}

Producing good tables and referencing these tables within a R Markdown
PDF has been a hassle but got much better. Examples that you may use are
shown below. The way you reference tables is slightly different, e.g.,
for \texttt{stargazer} the label is contained in the function, for
\texttt{kable} it's contained in the chunk name.

\hypertarget{stargazer-summary-and-regression-tables}{%
\subsection{stargazer(): Summary and regression
tables}\label{stargazer-summary-and-regression-tables}}

Table @ref(tab1) shows summary stats of your data.\footnote{To reference
  the table where you set the identifier in the stargazer function you
  only need to use the actual label, i.e., ´tab1´.} I normally use
\texttt{stargazer()} {[}@hlavac2013stargazer{]} which offers extreme
flexibility regarding table output (see \texttt{?stargazer}).

\begin{Shaded}
\begin{Highlighting}[]
\FunctionTok{library}\NormalTok{(stargazer)}
\FunctionTok{stargazer}\NormalTok{(cars, }
          \AttributeTok{title =} \StringTok{"Summary table with stargazer"}\NormalTok{,}
          \AttributeTok{label=}\StringTok{"tab1"}\NormalTok{, }
          \AttributeTok{table.placement =} \StringTok{"H"}\NormalTok{, }
          \AttributeTok{header=}\ConstantTok{FALSE}\NormalTok{)}
\end{Highlighting}
\end{Shaded}

\begin{table}[H] \centering 
  \caption{Summary table with stargazer} 
  \label{tab1} 
\begin{tabular}{@{\extracolsep{5pt}}lccccc} 
\\[-1.8ex]\hline 
\hline \\[-1.8ex] 
Statistic & \multicolumn{1}{c}{N} & \multicolumn{1}{c}{Mean} & \multicolumn{1}{c}{St. Dev.} & \multicolumn{1}{c}{Min} & \multicolumn{1}{c}{Max} \\ 
\hline \\[-1.8ex] 
speed & 50 & 15.400 & 5.288 & 4 & 25 \\ 
dist & 50 & 42.980 & 25.769 & 2 & 120 \\ 
\hline \\[-1.8ex] 
\end{tabular} 
\end{table}

Table @ref(tab2) shows the output for a regression table. Make sure you
name all your models and explicitly refer to model names (M1, M2 etc.)
in the text.

\begin{Shaded}
\begin{Highlighting}[]
\FunctionTok{library}\NormalTok{(stargazer)}
\NormalTok{model1 }\OtherTok{\textless{}{-}} \FunctionTok{lm}\NormalTok{(speed }\SpecialCharTok{\textasciitilde{}}\NormalTok{ dist, }\AttributeTok{data =}\NormalTok{ cars)}
\NormalTok{model2 }\OtherTok{\textless{}{-}} \FunctionTok{lm}\NormalTok{(speed }\SpecialCharTok{\textasciitilde{}}\NormalTok{ dist, }\AttributeTok{data =}\NormalTok{ cars)}
\NormalTok{model3 }\OtherTok{\textless{}{-}} \FunctionTok{lm}\NormalTok{(dist }\SpecialCharTok{\textasciitilde{}}\NormalTok{ speed, }\AttributeTok{data =}\NormalTok{ cars)}
\FunctionTok{stargazer}\NormalTok{(model1, model2, model3,}
          \AttributeTok{title =} \StringTok{"Regression table with stargazer"}\NormalTok{,}
          \AttributeTok{label=}\StringTok{"tab2"}\NormalTok{, }
          \AttributeTok{table.placement =} \StringTok{"H"}\NormalTok{, }
          \AttributeTok{column.labels =} \FunctionTok{c}\NormalTok{(}\StringTok{"M1"}\NormalTok{, }\StringTok{"M2"}\NormalTok{, }\StringTok{"M3"}\NormalTok{),}
          \AttributeTok{model.numbers =} \ConstantTok{FALSE}\NormalTok{,}
          \AttributeTok{header=}\ConstantTok{FALSE}\NormalTok{)}
\end{Highlighting}
\end{Shaded}

\begin{table}[H] \centering 
  \caption{Regression table with stargazer} 
  \label{tab2} 
\begin{tabular}{@{\extracolsep{5pt}}lccc} 
\\[-1.8ex]\hline 
\hline \\[-1.8ex] 
 & \multicolumn{3}{c}{\textit{Dependent variable:}} \\ 
\cline{2-4} 
\\[-1.8ex] & \multicolumn{2}{c}{speed} & dist \\ 
 & M1 & M2 & M3 \\ 
\hline \\[-1.8ex] 
 dist & 0.166$^{***}$ & 0.166$^{***}$ &  \\ 
  & (0.017) & (0.017) &  \\ 
  & & & \\ 
 speed &  &  & 3.932$^{***}$ \\ 
  &  &  & (0.416) \\ 
  & & & \\ 
 Constant & 8.284$^{***}$ & 8.284$^{***}$ & $-$17.579$^{**}$ \\ 
  & (0.874) & (0.874) & (6.758) \\ 
  & & & \\ 
\hline \\[-1.8ex] 
Observations & 50 & 50 & 50 \\ 
R$^{2}$ & 0.651 & 0.651 & 0.651 \\ 
Adjusted R$^{2}$ & 0.644 & 0.644 & 0.644 \\ 
Residual Std. Error (df = 48) & 3.156 & 3.156 & 15.380 \\ 
F Statistic (df = 1; 48) & 89.567$^{***}$ & 89.567$^{***}$ & 89.567$^{***}$ \\ 
\hline 
\hline \\[-1.8ex] 
\textit{Note:}  & \multicolumn{3}{r}{$^{*}$p$<$0.1; $^{**}$p$<$0.05; $^{***}$p$<$0.01} \\ 
\end{tabular} 
\end{table}

\end{document}
